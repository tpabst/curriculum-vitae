\begin{rubric}{F O R M A T I O N}
%
%%%%%%%%%%%%%%%%%%%%%%%%%%%%%%%%%%%%%%%
\subrubric{\'{E}tudes}
%
\entry*[2009~--~2012]
  Master {\textbf{\emph{M2I}}} ({\emph{Manager en Ing�nierie Informatique}})
  � \href{http://www.itin.fr}{\textbf{\emph{l'ITIN}}}, 

  Promu \textbf{major de promotion} pour l'option \textbf{\emph{Ing�nierie des jeux vid�o}}
  Cergy (95).
  \entry*
  Fili�re de sp�cialit� M1 (\textbf{\emph{G�nie des logiciels embarqu�s}})
  \entry*
  Fili�re de sp�cialit� M2 (\textbf{\emph{Ing�nierie des jeux vid�o}})

\entry*[2007~--~2009]
   {\textbf{\emph{BTS IRIS}}} 
   ({\emph{Informatique et R�seaux pour l'Industrie et les Services techniques}})
   aux {\textbf{\emph{CFAI}}}
  \href{http://www.cfai-alsace.fr}{\emph{Centre de Formation d'Apprentis de l'Industrie}},
 Eckbolsheim (67).

\entry*[2005~--~2007]
    {\em L1 - Licence Physique et Application}, 
  � l'\href{http://www.unistra.fr}{\em Universit� Louis Pasteur},
  Strasbourg (67).

\entry*[Juin~2005]
  {\em Baccalaur�at g�n�ral}, s�rie scientifique, 
  au \href{http://schuman.ovh/pic/}{\em Lyc�e Robert Schuman},
  Haguenau (67).

\subrubric{Formations professionnelles}
\entry*[2009] S�minaire {\textbf{\emph{ORSYS}}} sur les techniques des r�saux �mergents (\emph{3 jours})
\entry*[2008] Formation Builder C++ dispens� {\textbf{\emph{ACI}}} (68) au sein de la soci�t� 
	 {\textbf{\emph{SIEMENS}}} (\emph{5 jours})
\end{rubric}
